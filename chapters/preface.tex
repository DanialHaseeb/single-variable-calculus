\documentclass[../main.tex]{subfiles}

\begin{document}
\chapter{Preface}
Before Newton ever formulated a description of universal gravitation 
or his famous laws of motion that changed physics forever, 
he did something arguably more important; changing mathematics forever too. 
I am talking, of course, about infinitesimal calculus.

Much like ourselves, Isaac Newton had to live through a deadly pandemic – 
the Bubonic plague. Much like today, universities were closed, 
students were sent home, and people were told to self-isolate. 
It was sometime around 1665, 
while he was in quarantine at his home in London, that he 
figured out a way to analyse the motion of objects using infinitesimals – 
that taking smaller and smaller increments of time give a more and more 
accurate calculation for an object’s speed at a particular instant. 
He was thinking in terms of distance and speed at the time, 
but these same ideas would later be generalised to apply to all sorts of 
functions in mathematics and physics – what we now call 
differential calculus.

Incredibly, at around exactly the same time, a German mathematician, 
Gottfried Wilhelm Leibniz, was also thinking about and developing 
a different branch of calculus, and would soon arrive at the same 
fundamentals of calculus as Newton, completely independently!

Leibniz, working at Leipzig University (my own alma mater), 
was thinking about the area under an arbitrary curve, 
and how it can be approximated using smaller and smaller shapes. 
He was deriving what we now call integral calculus, 
and much of the notation we use comes directly from him.

Newton and Leibniz would later have many disputes 
over who came up with calculus first; both claiming one copied the other. 
It is somewhat surprising and hard to believe at first that 
two people would come up with such a novel idea at exactly the same time 
in history, but all retrospective analysis shows that this was indeed a 
coincidence. The invention (or discovery?) of this magical branch 
of mathematics was inevitable – and the world was ready for it in 1665.

Not long after, the ideas of calculus would lead to 
the mathematical descriptions of gravity by Newton, 
and of electromagnetism by James Clerk Maxwell. 
These ideas sparked the first industrial revolution in England. 
They gave us the manufacturing capabilities to wage world wars. 
Electricity dramatically increased the quality of life for 
billions of people. A few centuries later, 
calculus put the first human being on the Moon and 
helps epidemiologists fight off the pandemics that plague us today.
\end{document}