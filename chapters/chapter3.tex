\documentclass[../main.tex]{subfiles}

\begin{document}
\chapter{Derivative Formulæ}
\section{General Functions}
For any two functions $u$ and $v$:
\[ (u + v)' = u' + v' \]
When there is a constant, $c$:
\[ (c \cdot u)' = c \cdot u' \]
\section{Trigonometric Functions}
\begin{align*}
    \frac{d}{dx} \sin x &= \cos x\\
    \frac{d}{dx} \cos x &= - \sin x
\end{align*}
\section{Product Rule}
\begin{equation}
    (u \cdot v)' = v \cdot u' + u \cdot v'
    \label{eqn:product-rule}
\end{equation}
\begin{exmp}
    To differentiate $f(x) = x^3 \sin x$, we let 
    $u = x^3$ and $v = \sin x$:
    \begin{align*}
        \therefore\quad u' &= 3x^2\\
                        v' &= \cos x
    \end{align*}
    From Equation \ref{eqn:product-rule}, we know:
    \begin{align*}
        f'(x)   &= vu' + uv'\\
                &= 3x^2 \sin x + x^3 \cos x
    \end{align*}
\end{exmp}
\end{document}