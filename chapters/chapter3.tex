\documentclass[../main.tex]{subfiles}

\begin{document}
\chapter{Derivative Formulæ}
\section{General Functions}
For any two functions $u$ and $v$:
\[ (u + v)' = u' + v' \]
When there is a constant, $c$:
\[ (c \cdot u)' = c \cdot u' \]
\section{Trigonometric Functions}
\begin{align*}
    \frac{d}{dx} \sin x &= \cos x\\
    \frac{d}{dx} \cos x &= - \sin x
\end{align*}
\section{Product Rule}
\begin{equation}
    (u \cdot v)' = v \cdot u' + u \cdot v'
    \label{eqn:product-rule}
\end{equation}
\begin{exmp}
    To differentiate $f(x) = x^3 \sin x$, we let 
    $u = x^3$ and $v = \sin x$:
    \begin{align*}
        \therefore \quad u' &= 3x^2\\
                         v' &= \cos x
    \end{align*}
    From Equation \ref{eqn:product-rule}, we know:
    \begin{align*}
        f'(x)   &= vu' + uv'\\
                &= 3x^2 \sin x + x^3 \cos x
    \end{align*}
\end{exmp}
\section{Quotient Rule}
\[ \left( \frac{u}{v} \right)' = \frac{vu' - uv'}{v^2} \]
\section{Chain Rule}
The \emph{chain rule} (in Leibniz's notation) can be written 
in the following way:
\[ \frac{dy}{dt} = \frac{dy}{dx} \cdot \frac{dx}{dt} \]
\begin{exmp}
    To differentiate $y = \sin^{10} t$, we let 
    $x = \sin t$:
    \begin{align*}
        y &= x^{10}\\
        \therefore \quad \frac{dy}{dx} &= 10x^9
    \end{align*}
    From the chain rule, we get:
    \[ \frac{dy}{dt} = 10x^9 \cos t \quad \because \quad \frac{dx}{dt} = \cos t \]
    Finally, we need to substitute $x = \sin t$:
    \[ \frac{dy}{dt} = 10 \sin^9 t \cos t \]
\end{exmp}
\begin{exmp}
    To differentiate $\sin (10t)$, we let 
    $x = 10t$ and $y = \sin x$:
    We now have:
    \begin{align*}
        \frac{dx}{dt} &= 10\\
        \frac{dy}{dx} &= \cos x\\
        \frac{d}{dt} \sin (10t) &= \frac{dy}{dt}\\
        \therefore \quad \frac{dy}{dt}  &= \frac{dy}{dx} \cdot \frac{dx}{dt}\\
                                        &= 10 \cos x\\
                                        &= 10 \cos (10t)
    \end{align*}
\end{exmp}
\end{document}