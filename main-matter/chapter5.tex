\documentclass[../main]{subfiles}

\begin{document}
%
\chapter{Implicit Differentiation}
%--------------------------------------------------
We know:
\[ \frac{d}{dx} x^a = ax^{a-1} \quad \mid \quad a \in \mathbb{Z} \]
We will now extend this formula to cover $\mathbb{Q}$ as well:
\[
    a = \frac{m}{n} \rightarrow y = x^{\frac{m}{n}}
    \quad \mid \quad
    m, n \in \mathbb{Z}
\]
We can start computing the derivative using the chain rule:%
%
\begin{align*}
    y^n                                                 & = x^n                             \\
    \frac{d}{dx} \left( y^n \right)                     & = \frac{d}{dx} \left( x^m \right) \\
    \frac{d}{dy} \left( y^n \right) \cdot \frac{dy}{dx} & = mx^{m - 1}                      \\
    \frac{dy}{dx} \left( ny^{n - 1} \right)             & = mx^{m - 1}
\end{align*}
%
We finally have an expression for $y'$:
%
\begin{align*}
    \frac{dy}{dx}               & = \frac{mx^{m - 1}}{ny^{n - 1}}                                              \\
                                & = \frac{m}{n} \cdot \frac{x^{m - 1}}{y^{n - 1}}                              \\
                                & = \frac{m}{n} \cdot \frac{x^{m - 1}}{\left( x^{\frac{m}{n}} \right)^{n - 1}} \\
                                & = ax^{m - 1 - \frac{m}{n}(n - 1)}                                            \\
                                & = ax^{a - 1}                                                                 \\
    \therefore \frac{d}{dx} x^a & = ax^{a - 1} \quad \mid \quad a \in \mathbb{Q}
\end{align*}
%
\begin{exmp}
    The equation of a unit circle is:
    \[ y^2 = 1 - x^2 \]
    This can be rewritten as:
    \[ y = \left( 1 - x^2 \right)^{\frac{1}{2}} \]
    We can compute the derivative using the chain rule:
    %
    \begin{align*}
        \frac{dy}{dx} & = \frac{1}{2} \left( 1 - x^2 \right)^{\frac{1}{2} - 1} \cdot (-2x) \\
                      & = \frac{-x}{\left( 1 - x^2 \right)^{\frac{1}{2}}}                  \\
                      & = - \frac{x}{y}
    \end{align*}
    %
    However, we can do the same thing using \emph{implicit differentiation}:
    %
    \begin{align*}
        x^2 + y^2                                                         & = 1                \\
        \frac{d}{dx} \left( x^2 + y^2 \right)                             & = \frac{d}{dx} (1) \\
        \frac{d}{dx} \left( x^2 \right) + \frac{d}{dx} \left( y^2 \right) & = 0                \\
        2x + \frac{d}{dy} \left( y^2 \right) \cdot \frac{dy}{dx}          & = 0                \\
        2x + 2y y'                                                        & = 0                \\
        y'                                                                & = - \frac{x}{y}
    \end{align*}
    %
\end{exmp}
%
\begin{exmp}
    In the following case, it is not so easy to solve for $y$:
    \[ y^3 +xy^2 + 1 = 0 \]
    We will need to use implicit differentiation to find the derivative:
    %
    \begin{align*}
        3y^2y' + y^2 + 2xyy' + 0     & = 0                        \\
        y' \left( 3y^2 + 2xy \right) & = -y^2                     \\
        y'                           & = - \frac{y^2}{3y^2 + 2xy}
    \end{align*}
    %
\end{exmp}
%--------------------------------------------------
\section{Inverses}
%
If $y = f(x)$ and $g(y) = x$, we call $g$ the \emph{inverse} of $f$,
denoted $f^{-1}$:
\[ x = g(y) = f^{-1}(y) \]
Now, we will use implicit differentiation to find the derivative
of the inverse function:
%
\begin{align*}
    y                                                         & = f(x)                    \\
    f^{-1}(y)                                                 & = x                       \\
    \frac{d}{dx} \left( f^{-1}(y) \right)                     & = \frac{d}{dx}(x)         \\
    \frac{d}{dy} \left( f^{-1}(y) \right) \cdot \frac{dy}{dx} & = 1                       \\
    \frac{d}{dy} \left( f^{-1}(y) \right)                     & = \frac{1}{\frac{dy}{dx}} \\
\end{align*}
%
\begin{exmp}
    The derivative of $y = \tan^{-1}(x)$:
    %
    \begin{align*}
        \tan y                                                 & = x                                  \\
        \frac{d}{dx} \left( \tan y \right)                     & = \frac{d}{dx}(x)                    \\
        \frac{d}{dy} \left( \tan y \right) \cdot \frac{dy}{dx} & = 1                                  \\
        \left( \csc^2 y \right) \cdot \frac{dy}{dx}            & = 1                                  \\
        \frac{dy}{dx}                                          & = \cos^2 y                           \\
                                                               & = \cos^2 \left( \tan^{-1}(x) \right)
    \end{align*}
    %
    This form is messy but we can use geometry to simplify.
    %
    \begin{figure}[ht]
        \centering
        \begin{tikzpicture}[scale = 0.5 \textwidth / 2cm]
            \coordinate (A) at (0,0);
            \coordinate (B) at (2,0);
            \coordinate (C) at (0,1);
            \draw
            (A)
            --
            node[below = 5pt]
                {$x$}
            (B)
            --
            node
            [
            sloped,
            above = 5pt
            ]
            {$\left( 1 + x^2 \right)^{\frac{1}{2}}$}
            (C)
            --
            node[left = 5pt]
                {$1$}
            cycle
            pic
                [
                    draw = green!50!black,
                    fill = green!25,
                    angle radius = 1cm,
                    "$y$",
                    -stealth
                ]
                {angle = A--C--B}
            pic
                [
                    draw = black,
                    angle radius = 0.5cm
                ]
                {right angle = B--A--C};
        \end{tikzpicture}
        \caption
        {
            Triangle with angles and lengths corresponding to
            those in the example illustrating differentiation
            using the inverse function
        }
        \label{fig:simplificationTriangle}
    \end{figure}

    In the triangle in Figure \ref{fig:simplificationTriangle},
    $\tan y = x \Rightarrow y = \tan^{-1}(x)$. From this, we can find:
    \begin{align*}
        \cos y                                   & = \frac{1}{\sqrt{1 + x^2}}                  \\
        \left( \cos y \right)^2                  & = \left( \frac{1}{\sqrt{1 + x^2}} \right)^2 \\
        \cos^2 y                                 & = \frac{1}{1 + x^2}                         \\
        \frac{dy}{dx}                            & = \frac{1}{1 + x^2}                         \\
        \frac{d}{dx} \left( \tan^{-1}(x) \right) & = \frac{1}{1 + x^2}
    \end{align*}
    %
\end{exmp}
%
\subsection{Graphing}
%
Suppose $y = f(x)$ and $g(y) = f^{-1}(y) = x$.

To graph $f$ and $g$ together, we need to write $g$ as a function of $x$.
If $g(x) = y$, then $x = f(y)$. What we have done is trade the variables
$x$ and $y$. This is illustrated in Figure \ref{fig:inverseTangent}:
\begin{figure}[ht]
    \centering
    \begin{tikzpicture}
        [
            scale = \textwidth / 11cm,
            smooth,
            > = latex
        ]

        % Axes
        \draw[<->]  (-5,0) -- (5,0) node[right] {$x$};
        \draw[<->]  (0,-5) -- (0,5) node[above] {$y$};
        % Origin
        \node at (0,0) [below right = 2pt] {$0$};

        % Horizontal Aymptotes
        \draw
        [
            densely dotted,
            blue
        ]
        (-5,1.57)  -- (5,1.57);
        \draw
        [
            densely dotted,
            blue
        ]
        (-5,-1.57) -- (5,-1.57);
        % Vertical Asymptotes
        \draw[densely dotted] (1.57,-5)  -- (1.57,5);
        \draw[densely dotted] (-1.57,-5) -- (-1.57,5);

        % Ticks:
        \draw (-1.57,-2pt) -- (-1.57,2pt);
        \draw (1.57,-2pt)  -- (1.57,2pt);
        \draw (-2pt,-1.57) -- (2pt,-1.57);
        \draw (-2pt,1.57)  -- (2pt,1.57);

        % y = ±τ/4
        \node at (0,1.57)
        [
            right = 2pt,
            fill = white
        ]
        {$\frac{\tau}{4}$};
        \node at (0,-1.57)
        [
            right = 2pt,
            fill = white
        ]
        {$- \frac{\tau}{4}$};

        % x = ±τ/4
        \node at (-1.57,0)
        [
            below = 2pt,
            fill = white
        ]
        {$ - \frac{\tau}{4}$};
        \node at (1.57,0)
        [
            below = 2pt,
            fill = white
        ]
        {$ \frac{\tau}{4}$};

        % y = x
        \draw
        [
            gray,
            densely dashed
        ]
        plot[domain = -4 : 4]
        (\x,\x)
        node
            [
                right = 2pt,
                fill = white
            ]
            {$y = x$};

        % y = tan(x)
        \draw
        plot[domain = -1.37 : 1.37]
        (\x,{tan(\x r)}) % \x r means to convert '\x' from degrees to _r_adians:
        node
            [
                right = 2pt,
                fill = white
            ]
            {$y = \tan(x)$};

        % y = arctan(x)
        \draw[blue]
        plot[domain = -4.9 : 4.9]
        (\x,{rad(atan(\x))})
        node[below]
            {$y = \tan^{-1}(x)$};

        % Origin Mark:
        \filldraw[black] (0,0) circle [radius=1pt];
    \end{tikzpicture}
    \caption
    {
        We can think about $f^{-1}$ as the graph of $f$ reflected
        about the line $y = x$
    }
    \label{fig:inverseTangent}
\end{figure}
\end{document}